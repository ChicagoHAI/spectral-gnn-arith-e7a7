\begin{abstract}
We present a novel theoretical framework for approximating arbitrary arithmetic functions using graph neural networks with spectral filters. Our main result establishes that for any arithmetic function $f: \mathbb{R}^n \to \mathbb{R}$ composed of elementary operations, there exists a graph neural network architecture that can approximate $f$ with arbitrary precision while maintaining numerical stability. Specifically, we prove that for any $\epsilon > 0$ and bound $M > 0$, there exists a graph neural network $\mathcal{G}$ with $K$ spectral filters such that $\|f(x) - \mathcal{G}(x)\| \leq \epsilon$ for all $x \in [-M,M]^n$, with computational complexity $O(K\log(1/\epsilon))$ and condition number bounded by $C\log(1/\epsilon)$. The proof technique involves constructing a directed acyclic graph representing the arithmetic computation and leveraging Chebyshev polynomial approximations of the graph Laplacian spectrum. We develop novel bounds for elementary operations using spectral approximation theory and establish stability through careful analysis of filter coefficients. Our results provide theoretical guarantees for implementing arithmetic computations using graph neural networks while maintaining both accuracy and numerical stability, with potential applications in scientific computing and automated reasoning systems.
\end{abstract}