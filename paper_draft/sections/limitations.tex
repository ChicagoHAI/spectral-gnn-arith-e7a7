\section{Limitations and Future Work}

While our theoretical framework provides important guarantees for approximating arithmetic functions using spectral graph neural networks, several key limitations must be acknowledged:

\subsection{Domain Restrictions}
Our approximation guarantees hold only on compact domains of the form $[-M,M]^n$ for some $M > 0$. The behavior of our constructed networks for inputs outside this bounded domain remains theoretically uncharacterized. Specifically:

\begin{itemize}
\item The spectral radius bounds established in Lemma 1 may not hold for unbounded domains
\item The approximation error $\epsilon(x)$ could grow arbitrarily large as $\|x\| \to \infty$
\item Boundary effects near $x = \pm M$ may lead to degraded performance
\end{itemize}

\subsection{Numerical Considerations}
The practical implementation of our framework faces several numerical challenges:

\begin{itemize}
\item Computing the spectral decomposition of the graph Laplacian $L$ becomes numerically unstable for large graphs
\item Round-off errors in floating-point arithmetic may accumulate, particularly when evaluating the spectral filters $g_\theta(\lambda_i)$
\item The condition number of the system grows as $O(M^2)$, potentially leading to numerical instability for large domain bounds
\end{itemize}

\subsection{Computational Complexity}
The construction of our approximating networks requires:
\[
$O(|V|^3 + |E| \\log |V|)$
\]
operations for the spectral decomposition and filter computation. This may become prohibitive for large arithmetic expression graphs.

Future work should address these limitations by:
\begin{itemize}
\item Extending the theory to unbounded domains through appropriate scaling techniques
\item Developing numerically stable algorithms for computing spectral filters
\item Investigating sparse approximations to reduce computational complexity
\item Characterizing the trade-off between approximation accuracy and network size
\end{itemize}