\section{Background and Related Work}

The mathematical foundations for automated discovery of mathematical models and theorems span several interconnected domains. We review the key theoretical underpinnings and recent advances that inform our work.

\subsection{Neural Network Function Approximation}

The universal approximation theorem establishes that sufficiently wide neural networks can approximate any continuous function on compact subsets of $\mathbb{R}^n$. Recent work by \cite{testolin2023neural} has examined the precise capabilities of neural architectures in performing elementary arithmetic operations, providing rigorous bounds on approximation errors for basic computational tasks. Specifically, for an arithmetic function $f:\mathbb{R}^n \to \mathbb{R}$, they showed that a neural network with width $O(\log(1/\epsilon))$ can achieve approximation error $\leq \epsilon$ on compact domains.

\subsection{Spectral Graph Neural Networks}

The theoretical framework for spectral graph neural networks, as developed in \cite{mo2024autosgnn}, builds on the mathematical foundation of graph Laplacians $L = D - A$, where $D$ is the degree matrix and $A$ is the adjacency matrix. The eigendecomposition $L = U\Lambda U^T$ provides the basis for spectral convolution operations:

\[
g_\theta \star x = U g_\theta(\Lambda) $U^T$ x
\]

where $g_\theta$ is a learnable filter function. \cite{hosseini2022deep} demonstrated the effectiveness of such spectral approaches in automated discovery tasks, particularly for structured domains.

\subsection{Constitutive Model Discovery}

A significant advancement in automated mathematical discovery comes from \cite{linka2022automated} and \cite{linka2022family}, who developed frameworks for discovering constitutive relationships in complex systems. Their approach uses specialized neural architectures that incorporate physical constraints through carefully designed loss functions:

\[
\mathcal{L}(\theta) = \mathcal{L}_\text{data}(\theta) + \lambda \mathcal{L}_\text{physics}(\theta)
\]

This framework has been extended to various domains, as demonstrated by \cite{wang2022automated} and \cite{shobanadevi2024automated} in their respective applications.

\subsection{Automated Theorem Discovery}

The theoretical foundation for automated theorem discovery, as detailed by \cite{pantsar2024theorem}, relies on encoding logical statements as vectors in high-dimensional spaces. Given a logical formula $\phi$, the encoding function $E:\text{Formula} \to \mathbb{R}^d$ must satisfy:

\[
\text{Valid}(\phi) \implies \|E(\phi)\| \leq \delta
\]

for some small $\delta > 0$. Recent work by \cite{ma2022analysis} and \cite{yu2022data} has extended these ideas to pattern discovery in structured domains.

\subsection{Sequential Pattern Mining}

The temporal aspects of mathematical discovery are addressed through sequential pattern mining approaches. \cite{lee2019clustering} and \cite{ding2022prefixpruningbased} developed efficient algorithms for discovering temporal patterns, while \cite{wang2020improved} introduced optimization techniques for high-utility pattern mining. These approaches have been integrated with distributed computing frameworks by \cite{gabidullina2025data} and \cite{pandey2017developing}, enabling scalable pattern discovery.

\subsection{Applications to Complex Systems}

The practical application of these theoretical frameworks has been demonstrated in various domains. \cite{qashou2022mining} applied temporal pattern mining to complex system analysis, while \cite{yu2022data} showed how mathematical model discovery can be automated in biological systems. These applications validate the broader applicability of automated mathematical discovery techniques.

This rich theoretical foundation provides the basis for our work in developing novel approaches to automated mathematical discovery, building particularly on the spectral graph neural network framework of \cite{mo2024autosgnn} and the constitutive model discovery techniques of \cite{linka2022automated}.